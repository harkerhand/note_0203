\documentclass[a4paper,12pt]{article}
\usepackage{ctex} % 中文支持
\usepackage{listings} % 代码排版
\usepackage{xcolor}
\usepackage{graphicx} % 插入图片
\usepackage{geometry} % 页面设置
\usepackage{caption} % 图片标题

\geometry{margin=2.5cm}

% 代码风格设置
\lstset{
    language=Python, % 可以根据需要改成 C++, Java 等
    numbers=left,
    numberstyle=\tiny\color{gray},
    keywordstyle=\color{blue},
    commentstyle=\color{green!60!black},
    stringstyle=\color{orange},
    basicstyle=\ttfamily\small,
    breaklines=true,
    frame=single
}

\title{人工智能概论第二次实验报告}
\author{09023321 \ 巩皓锴}


\begin{document}

\maketitle

\section{实验内容}
    使用 A* 算法求解24数码问题

\section{实验代码}

\begin{lstlisting}[language=Python, caption=A*算法代码示例]
    import heapq
    # 定义曼哈顿距离启发函数
    def manhattan_distance(state, goal):
        distance = 0
        # 创建数字位置映射表
        goal_positions = {}
        for i in range(5):
            for j in range(5):
                goal_positions[goal[i][j]] = (i, j)
        
        # 计算每个数字的曼哈顿距离之和
        for i in range(5):
            for j in range(5):
                if state[i][j] != 0:  # 跳过空白格
                    goal_i, goal_j = goal_positions[state[i][j]]
                    distance += abs(i - goal_i) + abs(j - goal_j)
        return distance
    
    # 定义方向移动
    directions = {
        'UP': (-1, 0),
        'DOWN': (1, 0),
        'LEFT': (0, -1),
        'RIGHT': (0, 1)
    }
    
    # 查找空白格位置
    def find_blank(state):
        for i in range(5):
            for j in range(5):
                if state[i][j] == 0:
                    return i, j
        return -1, -1
    
    # 生成下一个状态
    def get_next_states(state, blank_x, blank_y):
        next_states = []
        for dir_name, (dx, dy) in directions.items():
            new_x, new_y = blank_x + dx, blank_y + dy
            if 0 <= new_x < 5 and 0 <= new_y < 5:
                new_state = copy.deepcopy(state)
                new_state[blank_x][blank_y], new_state[new_x][new_y] = new_state[new_x][new_y], new_state[blank_x][blank_y]
                next_states.append((new_state, dir_name, (new_x, new_y)))
        return next_states
    
    # 将状态转为元组,便于哈希
    def state_to_tuple(state):
        return tuple(tuple(row) for row in state)
    
    # 开始A*搜索
    start_blank_x, start_blank_y = find_blank(start_state)
    initial_h = manhattan_distance(start_state, goal_state)
    
    # 优先队列(f值, g值, 状态, 空白格坐标, 路径)
    open_list = [(initial_h, 0, start_state, (start_blank_x, start_blank_y), [])]
    closed_set = set()  # 已访问状态集合
    
    while open_list:
        # 取出f值最小的状态
        f, g, current_state, (blank_x, blank_y), path = heapq.heappop(open_list)
        
        # 状态转元组用于哈希检查
        state_tuple = state_to_tuple(current_state)
        
        # 如果是目标状态,返回路径
        if state_tuple == state_to_tuple(goal_state):
            return path
        
        # 如果已经访问过此状态,则跳过
        if state_tuple in closed_set:
            continue
        
        # 标记为已访问
        closed_set.add(state_tuple)
        
        # 生成所有可能的下一步状态
        for next_state, direction, (new_blank_x, new_blank_y) in get_next_states(current_state, blank_x, blank_y):
            next_state_tuple = state_to_tuple(next_state)
            
            # 如果该状态已访问过,则跳过
            if next_state_tuple in closed_set:
                continue
            
            # 计算新的g值和h值
            new_g = g + 1
            new_h = manhattan_distance(next_state, goal_state)
            new_f = new_g + new_h
            
            # 添加到开放列表
            heapq.heappush(open_list, (new_f, new_g, next_state, (new_blank_x, new_blank_y), path + [direction]))
    
    # 如果无解,返回空列表
    return moves
\end{lstlisting}


\section{实验结果}
\subsection{运行结果}

\begin{lstlisting}
    生成的初始状态:
[' 1', ' 2', ' 3', ' 4', ' 5']
['11', '13', ' 6', ' 8', '10']
[' 0', '12', ' 7', ' 9', '15']
['16', '17', '18', '24', '19']
['21', '22', '23', '20', '14']

可解性验证: 通过

参考解步数: 30
参考解: ['LEFT', 'LEFT', 'DOWN', 'RIGHT', 'UP', 'LEFT', 'DOWN', 'RIGHT', 'UP', 'RIGHT', 'DOWN', 'LEFT', 'LEFT', 'LEFT', 'UP', 'LEFT', 'UP', 'RIGHT', 'UP', 'LEFT', 'DOWN', 'RIGHT', 'UP', 'LEFT', 'DOWN', 'DOWN', 'RIGHT', 'UP', 'LEFT', 'UP']

运行A*算法验证...
实际找到解(22步)
解: ['RIGHT', 'UP', 'RIGHT', 'DOWN', 'LEFT', 'LEFT', 'UP', 'RIGHT', 'RIGHT', 'RIGHT', 'DOWN', 'RIGHT', 'DOWN', 'DOWN', 'LEFT', 'UP', 'RIGHT', 'UP', 'LEFT', 'DOWN', 'RIGHT', 'DOWN']
\end{lstlisting}

\subsection{结果分析}
A* 算法能够有效求解本实验中的24数码问题,并能在合理时间内找到相对较优的解。
解路径与步数均优于参考解,显示了该算法良好的性能。


\section*{四、实验分析}

本实验采用 A\* 算法求解24数码问题。A\* 算法的核心思想是结合实际代价 (g 值) 和预估代价 (h 值) 进行优先搜索,保证搜索路径既高效又接近最优。实验中,启发函数采用的是每个数字块到目标位置的曼哈顿距离之和,有效引导搜索方向。

搜索过程中,每一步选择当前代价最小的状态扩展,避免盲目遍历所有可能。通过合理设计启发函数,最终找到 22 步的解,比参考解更优,验证了 A\* 算法在本问题中的高效性。

\end{document}
